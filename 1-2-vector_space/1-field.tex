\documentclass[12pt]{article}
\usepackage[a4paper, total={17.18cm, 24.62cm}]{geometry}
\usepackage[onehalfspacing]{setspace}
\usepackage{amssymb}
\usepackage{amstext}
\usepackage{amsmath}

\begin{document}

\begin{center}
    Lan Yung-Chi, 2026/01/06
\end{center}

\section{Definition}

A \textit{field} \(\mathbf{F}\) is a set containing at least two distinct elements called 0 and 1, along with operations of addition and multiplication satifying all the properties listed below.

\begin{enumerate}
    \item \textbf{Commutativity}

    \hspace{5mm} \(\alpha + \beta = \beta + \alpha\) and \(\alpha \beta = \beta \alpha\), for all \(\alpha, \beta \in \mathbf{F}\).

    \item \textbf{Associativity}

    \hspace{5mm} \((\alpha + \beta) + \lambda = \alpha + (\beta + \lambda)\) and \((\alpha \beta) \lambda = \alpha (\beta \lambda)\), for all \(\alpha, \beta, \lambda \in \mathbf{F}\).

    \item \textbf{Identities}

    \hspace{5mm} \(\lambda + 0 = \lambda\) and \(1 \lambda = \lambda\), for all \(\lambda \in \mathbf{F}\).

    \item \textbf{Additive Inverse}

    \hspace{5mm} For every \(\alpha \in \mathbf{F}\), there exists a unique \(\beta \in \mathbf{F}\) such that \(\alpha + \beta = 0\).

    \item \textbf{Multiplicative Inverse}

    \hspace{5mm} For every \(\alpha \in \mathbf{F}\) with \(\alpha \neq 0\), there exists a unique \(\beta \in \mathbf{F}\) such that \(\alpha \beta = 1\).

    \item \textbf{Distributive Property}

    \hspace{5mm} \(\lambda(\alpha + \beta) = \lambda \alpha + \lambda \beta\), for all \(\lambda, \alpha, \beta \in \mathbf{F}\).
\end{enumerate}

\subsubsection*{Note}

In linear algebra we rarely deal with fields other than \(\mathbf{R}\) and \(\mathbf{C}\), but other fields do exist such as the set of rational numbers along with the usual operations of addition and multiplication, and the set \(\{0, 1\}\) with the usual operations of addition and multiplication except that \(1 + 1\) is defined to equal 0.

\newpage

\section{Exercise}

Prove that the set of \textit{complex numbers} with the following operations of addition and multiplication is a field. For the sake of simplicity, we can trivially say that \(\mathbf{R}\) is a field, and safely use all calculation techniques from \(\mathbf{R}\).

\subsection{Definition}

\begin{itemize}
    \item A complex number is an ordered pair \((a, b)\), where \(a, b \in \mathbf{R}\), written as \(a + b \,i\).
    \item The set of all complex numbers is denoted by \(\mathbf{C}\):
    \[
        \mathbf{C} = \{ a + b \,i \mid a, b \in \mathbf{R} \}.
    \]
    \item Addition and multiplication on \(\mathbf{C}\) are defined by
    \begin{equation}
        (a + b \,i) + (c + d \,i) = (a + c) + (b + d) \,i,
    \end{equation}
    \begin{equation}
        (a + b \,i)(c + d \,i) = (ac - bd) + (ad + bc) \,i;
    \end{equation}
    with \(a, b, c, d \in \mathbf{R}\).
\end{itemize}

\subsection{Proof}

\begin{enumerate}
    \item \textbf{Commutativity}

    For all \(\alpha, \beta \in \mathbf{C}\), we can write \(\alpha = a + b \,i\) and \(\beta = c + d \,i\) with some \(a, b, c, d \in \mathbf{R}\), then
    \begin{align*}
        \alpha + \beta
        &= (a + b \,i) + (c + d \,i)    &\\
        &= (a + c) + (b + d) \,i        &\text{by (1)}\\
        \beta + \alpha
        &= (c + d \,i) + (a + b \,i)    &\\
        &= (c + a) + (d + b) \,i        &\text{by (1)}\\
        &= \alpha + \beta               &\text{by comm. of } \mathbf{R}\\
        &                               &\\
        \alpha \beta
        &= (a + b \,i)(c + d \,i)       &\\
        &= (ac - bd) + (ad + bc) \,i    &\text{by (2)}\\
        \beta \alpha
        &= (c + d \,i)(a + b \,i)       &\\
        &= (ca - db) + (da + cb) \,i    &\text{by (2)}\\
        &= \alpha \beta                 &\text{by comm. of } \mathbf{R}
    \end{align*}

    \newpage

    \item \textbf{Associativity}

    For all \(\alpha, \beta, \lambda \in \mathbf{C}\), write \(\alpha = a + b \,i\), \(\beta = c + d \,i\), \(\lambda = e + f \,i\) with some \(a, b, c, d, e, f \in \mathbf{R}\), then
    \begin{align*}
        (\alpha + \beta) + \lambda
        &= [(a + b \,i) + (c + d \,i)] + (e + f \,i)    &\\
        &= [(a + c) + (b + d) \,i] + (e + f \,i)        &\text{by (1)}\\
        &= [(a + c) + e] + [(b + d) + f] \,i            &\text{by (1)}\\
        \alpha + (\beta + \lambda)
        &= (a + b \,i) + [(c + d \,i) + (e + f \,i)]    &\\
        &= (a + b \,i) + [(c + e) + (d + f) \,i]        &\text{by (1)}\\
        &= [a + (c + e)] + [b + (d + f)] \,i            &\text{by (1)}\\
        &= (\alpha + \beta) + \lambda                   &\text{by asso. of } \mathbf{R}\\
        \\
        (\alpha \beta) \lambda
        &= [(a + b \,i)(c + d \,i)](e + f \,i)          &\\
        &= [(ac - bd) + (ad + bc) \,i](e + f \,i)       &\text{by (2)}\\
        &= [(ac - bd)e - (ad + bc)f] + [(ac - bd)f + (ad + bc)e] \,i    &\text{by (2)}\\
        &= (ace - bde - adf - bcf) + (acf - bdf + ade + bce) \,i        &\text{by dist. of } \mathbf{R}\\
        \alpha (\beta \lambda)
        &= (a + b \,i)[(c + d \,i)(e + f \,i)]          &\\
        &= (a + b \,i)[(ce - df) + (cf + de) \,i]       &\text{by (2)}\\
        &= [a(ce - df) - b(cf + de)] + [a(cf + de) + b(ce - df)] \,i    &\text{by (2)}\\
        &= (ace - bde - adf - bcf) + (acf - bdf + ade + bce) \,i        &\text{by dist. of } \mathbf{R}\\
        &= (\alpha \beta) \lambda                       &\text{by comm. of } \mathbf{R}
    \end{align*}

    \item \textbf{Identities}

    For all \(\lambda \in \mathbf{C}\), write \(\lambda = a + b \,i\) for some \(a, b \in \mathbf{R}\), then
    \begin{align*}
        \lambda + 0
        &= (a + b \,i) + (0 + 0 \,i)    &\\
        &= (a + 0) + (b + 0) \,i        &\text{by (1)}\\
        &= a + b \,i                    &\text{by iden. of } \mathbf{R}\\
        &= \lambda                      &\\
        \\
        1 \lambda
        &= (1 + 0 \,i)(a + b \,i)       &\\
        &= (1a + 0b) + (1b + 0a) \,i    &\text{by (2)}\\
        &= a + b \,i                    &\text{by iden. and mul. of } \mathbf{R}\\
        &= \lambda
    \end{align*}

    \item \textbf{Additive Inverse}

    For every \(\alpha \in \mathbf{C}\), write \(\alpha = a + b \,i\) for some \(a, b \in \mathbf{R}\), then we can find a unique \(\beta = c + d \,i \in \mathbf{C}\) such that \(\alpha + \beta = 0\) by the following manner:
    \begin{align*}
        \alpha + \beta
        &= (a + b \,i) + (c + d \,i)    &\\
        &= (a + c) + (b + d) \,i        &\text{by (1)}\\
        &= 0 + 0 \,i                    &\text{by assumption}\\
        \Rightarrow (c, d) &= (-a, -b) &\text{by add.inv. of } \mathbf{R}
    \end{align*}
    Thus \(\beta\) is the unique additive inverse of \(\alpha\).

    \item \textbf{Multiplicative Inverse}

    For every \(\alpha \in \mathbf{C}\), write \(\alpha = a + b \,i\) for some \(a, b \in \mathbf{R}\) with \(a \neq 0\) and \(b \neq 0\), then we can find a unique \(\beta = c + d \,i \in \mathbf{C}\) such that \(\alpha \beta = 1\) by the following manner:
    \begin{align*}
        \alpha \beta
        &= (a + b \,i)(c + d \,i)       &\\
        &= (ac - bd) + (ad + bc) \,i    &\text{by (2)}\\
        &= 1 + 0 \,i                    &\text{by assumption}
    \end{align*}
    That is, \(ac - bd = 1\) and \(ad + bc = 0\). By solving these two equations for \(c\) and \(d\), we have
    \[
        (c, d) = \left(\frac{a}{a^2 + b^2}, \; \frac{-b}{a^2 + b^2}\right),
    \]
    thus \(\beta\) is the unique multiplicative inverse of \(\alpha\).

    \item \textbf{Distributive Property}
    
    For all \(\alpha, \beta, \lambda \in \mathbf{C}\), write \(\alpha = a + b \,i\), \(\beta = c + d \,i\), \(\lambda = e + f \,i\) with some \(a, b, c, d, e, f \in \mathbf{R}\), then
    \begin{align*}
        \lambda(\alpha + \beta)
        &= (e + f \,i)[(a + b \,i) + (c + d \,i)]               &\\
        &= (e + f \,i)[(a + c) + (b + d) \,i]                   &\text{by (1)}\\
        &= [e(a + c) - f(b + d)] + [e(b + d) + f(a + c)] \,i    &\text{by (2)}\\
        &= (ea + ec - fb - fd) + (eb + ed + fa + fc) \,i        &\text{by dist. of } \mathbf{R}\\
        \\
        \lambda \alpha + \lambda \beta
        &= [(e + f \,i)(a + b \,i)] + [(e + f \,i)(c + d \,i)]          &\\
        &= [(ea - fb) + (eb + fa) \,i] + [(ec - fd) + (ed + fc) \,i]    &\text{by (2)}\\
        &= (ea - fb + ec - fd) + (eb + fa + ed + fc) \,i                &\text{by (1)}\\
        &= \lambda(\alpha + \beta)                                      &\text{by comm. of } \mathbf{R}
    \end{align*}
\end{enumerate}

\end{document}